\documentclass{article}
\usepackage[utf8]{inputenc}
\usepackage{amsmath}
\usepackage{textcomp}
\usepackage{eurosym}


\title{Mathematical formulation of the pump scheduling probleme as a MINLP}
\author{Gratien Bonvin}
\date{Montréal, 22 janvier 2018}

\usepackage{natbib}
\usepackage{graphicx}

\begin{document}

\maketitle

\section{Notations and variables}
~~~ A water network can be mathematically described as a directed graph
$G=(N,A)$ :

~~ -the nodes N are divided into tanks nodes ${N}_{T}$, sources nodes ${N}_{S}$ and
~~ junctions nodes ${N}_{J}$

~~ -the arcs $A$ are either pumps $K$, valves $V$ or pipes $L$

Because both the water demand and electricity tariffs often fluctuate on
a daily basis, the scheduling horizon is most commonly limited to 24 hours.
This time interval is divided in $T$ periods of length $\tau_t$ (in hours) during
which we consider that
ows and pressures are in a steady state.

We define two sets of implied variables, the
flow rate qat (in $m^{3}/h$) and the
hydraulic head hnt (in $m$), respectively defined as the water
flowing through
arcs $a \in A$ and the sum of the geographical elevation and the water pressure
head at node $n \in N$ during period $t$. In the following, we summarize how to
model the physical assets and we define the decision variables.




\section{Modeling physical assets}
\subsection{Pipes}

~~~~~~ Friction head losses are well-approximated by the Hazen-Williams or
Darcy-Weisbach formulae. Because they are difficult to handle in a optimisation
framework, sufficiently accurate quadratic approximations have been
proposed [1, 2]. In order to facilitate the comparison with previous proposed
methods, we also adopt a quadratic approximation for head losses $\phi_l$,

\begin{equation}
\phi_l(q_{lt}) = A_lq_{lt}.\left | q_{lt} \right | + B_lq_{lt}
\end{equation}

\subsection{Pumps}

Following [3], the head increase of a pump $k \in K$ can be written as
\begin{equation}
\Psi_k(q_{kt},\omega_{kt}) = \omega_{kt}^{2}\left(C_k+D_k\left(\frac{q_{kt}}{\omega_{kt}}\right)^{\gamma_a}\right)
\end{equation}

and the power consumption as

\begin{equation}
\Gamma_k(q_{kt},\omega_{kt}) = \omega_{kt}^{3}\left(E_k+F_k\left(\frac{q_{kt}}{\omega_{kt}} \right ) \right )
\end{equation}

Here, $\omega_{kt}$ is the relative speed of the pump and pump parameters $C_k$, $D_k$, $E_k$
and $F_k$ are approximated from operating points provided by the manufacturer.
Notice that a pump is not always equipped with a variable speed
drive : in this case, it has to operate at nominal speed ($\omega_{kt} = 1$) and the power
consumption $\Gamma(q_{kt})$ becomes linear. Thus, we divide the set of pumps
$K$ into the subsets of fixed speed pumps $K_{FSP}$ and variable speed pumps
$K_{VSP}$.

Moreover, a pump can be switched on or off. This requires the definition
of a binary activity indicator $x_k$ which allows firstly to force the
flow and the
pump speed to be null when pump is off ($x_k = 0$) and in a suited positive
interval otherwise ($x_k = 1$) [4] and secondly, to decouple the head between
the pump end nodes when pump is off [5].

Finally, the head increase 	$\Psi_k(q_{kt})$ for fixed speed pumps is often approximated
by a quadratic approximation [6, 7, 8] and this is also the case for
the two case studies I send you\footnote{In the two case studies, we do not have variable speed pumps but I include them in
the model in order to have a general formulation.}. Thus, we adopt the same relation

\begin{equation}
\Psi_k(q_{kt}) = A_kq_{kt}^{2} + B_lq_{kt} + C_k
\end{equation}

\subsection{Valves}

~~~~ Different kind of valves are frequently installed in DWDNs, in particular
gate valves (GVs), check valves (CVs),
flow control valves (FCVs) or pressure
reducing valves (PRVs). Their respective purpose is to force the
flow to be
null when the valve is closed, to avoid negative
flows, to control the
flow
through the valve and to create a pressure decrease. For each valve type, a
valve $v \in V$ can be modeled with two equations,

\begin{equation}
Q_v^{min}g_v(x_{vt}) \leq q_{vt} \leq Q_v^{max}x_{vt}
\end{equation}

\begin{table}
  \centering
  \begin{tabular}{llll}
    \hline
  $A$ & arcs & $N$ & nodes\\
  $L\subset A$ & pipes & $N_R\subset N$ & tank nodes \\
  $V\subset A$ & valves & $N_S\subset N$ & source nodes\\
  $K\subset A$ &pumps& $N_J\subset N$ & junction nodes\\
  $K_{FSP}\subset K$ & fixed-speed pumps& $t\in[1,T]$ &time periods \\
  $K_{VSP}\subset K$ & variable-speed pumps & & \\[.5em]
  $x_{kt}\in\{0,1\}$ & \multicolumn{3}{l}{ activity indicator (binary) of pump $k \in K$ in period $t$}\\
  $x_{vt}\in\{0,1\}$ & \multicolumn{3}{l}{activity indicator of valve $v\in V$ in period $t$}\\
  $w_{kt}\in[0,1]$ & \multicolumn{3}{l}{speed of pump $k\in K_{VSP}$ in period $t$}\\
  $q_{at}\in \mathbb{R}$ & \multicolumn{3}{l}{flow through $a\in A$ in period $t$}\\
  $h_{nt}\ge 0$ & \multicolumn{3}{l}{hydraulic head at node $n\in N$ in period $t$}\\
    \hline
\end{tabular}
\caption{Summary of notation}
\label{tab:not}
\end{table}

\newpage

and

\begin{equation}
\Delta H_v^{min}(1-x_{vt}) \leq  h_{it}-h_{jt} \leq \Delta H_v^{max}(1-g_v(x_{vt}))
\end{equation}

with $Q_a^{min} \leq q_{vt} \leq Q_v^{max} , \Delta H_v^{min} \leq  h_{it}-h_{jt} \leq \Delta H_v^{max}$, $g$ a function which depends wheter the valve is either a $GV$ or a $CV$ $(g_v(x_{vt})=x_{vt})$ , or a $FCV$ or a $PRV$ $(g_v(x_{vt})=1-x_{vt})$ and $x_{vt}$ a binary variable which have a different meaning for each type of valve. For a $GV$, when the valve is closed $(x_v=0)$, flow is null and head at both end nodes are decoupled, whereas heads and flow are untouched when the valve is open $(x_v=1)$.
For a $CV$, with $\Delta H_v^{max} = Q_v^{min}=0$, either the valve is open $(x_{vt}=1$, $ h_{it}-h_{jt}=0 $ and $ q_{vt} \geq 0 )$ or the valve is closed $(x_{vt}=0$, $ h_{it}-h_{jt} \leq 0 $ and $ q_{vt} = 0 )$.
for a $PRV$, $x_{vt}$ denotes the direction of the flow and constraint 6 forces the pressure to decrease in the flow direction. Finally because the flow is controlled by applying a pressure decrease at the valve level, $FCVs$ are mathematically modeled as $PRVs$, "the difference between both valves types being their control variable" [9]. Notice that for $FCVs$ and $PRVs$, the two equations can be replaced by the non-convex inequality $q_{vt}(hit-hjt) \geq 0$.




\section{Mathematical formulation}
With the help of notations given Table 1, the Pump Scheduling Problem can be formulated as a MINLP :

\newpage



\begin{equation}
min \sum_{t=1}^{T} C_t^{E} \tau_t\sum_{k \in K} \omega_{kt}^3\left(E_k+F_k\left(\frac {q_{kt}}{\omega_{kt}}\right)    \right) + \sum_{t=1}^{T} \sum_{s \in N_S} \sum_{si \in A} C_s^W q_{si} \tau_t
\end{equation}

\begin{equation}
s.t.~~~~\sum_{ij \in A} q_{ijt} = \sum_{ji \in A} q_{jit} + D_{jt}, ~~~~~~~~~~~~~~~~~~~~~~~~~~~ \forall t,~~\forall j \in N_J~~~~
\end{equation}



\begin{equation}
Z_j \leq h_{ij} \leq Z_j + P^{max}, ~~~~~~~~~~~~~~~~~ \forall t,~~\forall j \in N_J, ~~D_j^0 > 0
\end{equation}




\begin{equation}
\sum_{ir \in A} q_{irt} - \sum_{ri \in A}q_{rit} = \frac {S_r}{\tau_t}(h_{rt}-h_{r(t-1)}), ~~~~~~~~ \forall t,~~\forall r \in N_R
\end{equation}

\begin{equation}
h_{r0} = \frac {V_r^0}{S_r} \leq h_{rT}, ~~~~~~~~~~~~~~~~~~~~~~~~~~~~~~~~~~~~~~~~~\forall r \in N_R
\end{equation}

\begin{equation}
Z_r + \frac{V_r^{min}}{S_r} \leq h_{rt} \leq Z_r + \frac{V_r^{max}}{S_r}, ~~~~~~~~~~~~~~~~~~ \forall t, \forall r \in N_R
\end{equation}

\begin{equation}
\sum_{t=1}^T \sum_{si \in A} q_{sit}\tau_t \leq V_s^{lim}, ~~~~~~~~~~~~~~~~~~~~~~~~~~~~~~~ \forall t, \forall s \in N_S
\end{equation}

\begin{equation}
h_{st} = H_{st},   ~~~~~~~~~~~~~~~~~~~~~~~~~~~~~~~~~~~~~~~~~~~~~~\forall t, \forall s \in N_S
\end{equation}

\begin{equation}
Q_k^{min}x_{kt} \leq q_{kt} \leq Q_k^{max}x_{kt},  ~~~~~~~~~~~~~~~~~~~~~~~~~ \forall t, \forall k \in K
\end{equation}

\begin{equation}
\omega_k^{min}x_{kt} \leq \omega_{kt} \leq x_{kt},   ~~~~~~~~~~~~~~~~~~~~~~~~~~~~ \forall t, \forall k \in K_{VSP}
\end{equation}

\begin{equation}
m_k(1-x_{ijt}) \leq h_{jt}-h_{it}-\Psi_k(q_{ijt},\omega_{ijt}) \leq M_k(1-x_{ijt}), ~~~~~~ \forall t, \forall ij \in K_{VSP}
\end{equation}

\begin{equation}
h_{it}-h_{jt} = A_lq_{ijt}\left |q_{ijt}\right |  + B_lq_{ijt} ~~~~~~~~~~~~~~~~~~~~~\forall t, \forall ij \in L
\end{equation}

\begin{equation}
Q_v^{min} g_v(x_{vt}) \leq q_{vt} \leq Q_v^{max}x_{vt}   ~~~~~~~~~~~~~~~~~~~~~\forall t, \forall v \in V
\end{equation}

\begin{equation}
\Delta H_v^{min}(1-x_{vt}) \leq h_{i_{v}t}-h_{jt} \leq \Delta H_v^{max}(1-g_v(x_{vt})), ~~~~~~~~ \forall t, \forall ij \in V
\end{equation}



\newpage

Objective (7) minimizes the total energy cost of pumping and the cost of extraction where $C_t$ denotes the electricity tariff (in \euro{}/kWh) and $C_j^W$ the extraction cost at sources (in \euro{}/$m^3$). Flow conservation is ensured by Contraints (8) at each internal node r \in $N_J$, $D_{rt}$ denotes the demand (in $m^3$) in time period t \in $[1,T]$ and $S_j$ the cross-section (in $m^2$) of the tank. Constraints (9), (11), (12) and (14) are decicated to head constraints in junctions, tanks and sources: for each junction j \in $N_J$, head have to be greater than its altitude $Z_j$ for node with positive demand in order to prevent nonnegative pressures for consumers\footnote{For the case Verleye, I have removed this constraint in order to match with the model described in [8]}. A maximal pressure $P^{max}$ (that I fix to 100 $m$) can also be defined for safety reasons. For each tank r \in $N_R$, the initial one water volume is fixed, we impose that the final water volume is greater than the initial in order to avoid disproportionately pumping costs the following day [3] and otherwise the minimal and maximal water level $V_j^{min}$ and $V_r^{max}$ is limited respectively by safety reasons ans the physical tank size. For each source s \in $N_S$, the head can be fixed at each time because we assume that its water level is not impacted by the operation of the water network due to its size[10]. Moreover, in some cases, the daily extraction at a source can be restricted to be below a threshold $V_s^{max}$ (contraint(13)). Constraint (15) and (16) guarantees that flow and speed lie in the intervals $[Q_k^{min},Q_k^{max}]$ and $[\omega_k^{min},1]$ only when the pump $k \in K$ is active and that flow and speed are null otherwise. Finally, constraints (17 - 20) model the head difference along respectively pumps, pipes and valves.


\section{How are organized the CSV files ?}

There is one CSV file that regroup all the Data for each instance.
















Data dictionnary:
\\
Binary variable representing the state of the Pump (1 if ON, 0 if OFF).
Binary variable representing the state of the Valve (1 if ON, 0 if OFF).

\end{document}
